\documentclass[11pt,a4paper,twocolumn]{article}
\usepackage[left=1.5cm, top=2.5cm, total={18cm, 25cm}]{geometry}
\usepackage[utf8]{inputenc}
\usepackage[IL2]{fontenc}
\usepackage[czech]{babel}
\usepackage{times}
\usepackage{amsmath}
\usepackage{amsthm}
\usepackage{amssymb}
\usepackage{mathtools}
\newtheorem{veta}{Věta}
\newtheorem{definice}{Definice}

\begin{document}
\begin{titlepage}
\date{}
\begin{center}
\Huge\textsc{Fakulta informačních technologií \\
Vysoké učení technické v~Brně} \\
\vspace{\stretch{0.377}}
\LARGE Typografie a~publikování -- 2. projekt \\
Sazba dokumentů a~matematických výrazů \\
\vspace{\stretch{0.618}}
\end{center}
{\Large{2021\hfill
Adrián Bobola (xbobol00)}}
\end{titlepage}

\section*{Úvod}
V~této úloze si vyzkoušíme sazbu titulní strany, matematických vzorců, prostředí a dalších textových struktur obvyk\-lých pro technicky zaměřené texty (například rovnice (\ref{rovnice1})
nebo Definice \ref{definice} na straně \pageref{definice}). Rovněž si vyzkoušíme používání odkazů \verb|\ref| a~\verb|\pageref|.

Na titulní straně je využito sázení nadpisu podle optického středu s~využitím zlatého řezu. Tento postup byl
probírán na přednášce. Dále je použito odřádkování se
zadanou relativní velikostí 0.4 em a~0.3 em.

V~případě, že budete potřebovat vyjádřit matematickou
konstrukci nebo symbol a nebude se Vám dařit jej nalézt
v~samotném \LaTeX u, doporučuji prostudovat možnosti balíku maker \AmS-\LaTeX.

\section{Matematický text}
Nejprve se podíváme na sázení matematických symbolů
a~výrazů v~plynulém textu včetně sazby definic a vět s~využitím balíku \verb|amsthm|. Rovněž použijeme poznámku pod
čarou s~použitím příkazu \verb|\footnote|. Někdy je vhodné
použít konstrukci \verb|\mbox{}|, která říká, že text nemá být
zalomen.

\begin{definice}\label{definice}
\textnormal{Rozšířený zásobníkový automat} \textit{(RZA) je definován jako sedmice tvaru $A= (Q, \Sigma, \Gamma, \delta, q_0, Z_0, F)$,
kde:}
\begin{itemize}
\item \textit{$Q$ je konečná množina \textnormal{vnitřních (řídicích) stavů,}}
\item \textit{$\Sigma$ je konečná \textnormal{vstupní abeceda,}}
\item \textit{$\Gamma$ je konečná \textnormal{zásobníková abeceda,}}
\item \textit{$\delta$ je \textnormal{přechodová funkce} 
$Q\times(\Sigma\cup\{\epsilon\})\times\Gamma^*\rightarrow2^{Q\times\Gamma^*}$,}
\item \textit{$q_0 \in Q$ je \textnormal{počáteční stav,} $Z_0 \in \Gamma$ je \textnormal{startovací symbol
zásobníku} a $F$ $\subseteq$ $Q$ je množina} \textnormal{koncových stavů.}
\end{itemize}
\end{definice}

Nechť \textit{$P = (Q, \Sigma, \Gamma, \delta, q_0, Z_0, F)$} je rozšířený zásob- níkový automat. \textit{Konfigurací} nazveme trojici $(q, w, \alpha)\in Q\times\Sigma^*\times\Gamma^*$, kde $q$ je aktuální stav vnitřního řízení,
$w$~je~dosud nezpracovaná část vstupního řetězce a $\alpha =$
$Z_{i_1}Z_{i_2}$~.~.~. $Z_{i_k}$ je obsah zásobníku\footnote{$Z_{i_1}$ je vrchol zásobníku}.

\subsection{Podsekce obsahující větu a~odkaz}
\begin{definice}\label{definice2}
\textnormal{Řetězec $w$ nad abecedou $\Sigma$ je přijat RZA}
A~jestliže $(q_0, w, Z_0) \underset{A}{\overset{*}{\vdash}} (q_F, \epsilon, \gamma)$ pro nějaké $\gamma$ $\in$  $\Gamma^{*}$ a 
$q_F \in F$. Množinu $L(A) = \textnormal{\{} w \mid w$ je přijat RZA A\textnormal{\}} $\subseteq$
$\Sigma^{*}$~\textit{nazýváme} \textnormal{jazyk přijímaný RZA }A.
\end{definice}

Nyní si vyzkoušíme sazbu vět a důkazů opět s~použitím
balíku \verb|amsthm|.
\begin{veta}Třída jazyků, které jsou přijímány ZA, odpovídá 
\textnormal{bezkontextovým jazykům.}\end{veta}
\begin{proof}
\textnormal{V~důkaze vyjdeme z~Definice \ref{definice} a \ref{definice2}.}
\end{proof}

\section{Rovnice a odkazy}
Složitější matematické formulace sázíme mimo plynulý
text. Lze umístit několik výrazů na jeden řádek, ale pak je
třeba tyto vhodně oddělit, například příkazem \verb|\quad|.

$$\sqrt[i]{x^{3}_{i}} \quad\text{kde $x_i$ je $i$-té sudé číslo splňující}
\quad x_{i}^{x_{i}^{i^{2}}+2} \leq y_{i}^{x_{i}^{4}}$$


V~rovnici (\ref{rovnice1}) jsou využity tři typy závorek s~různou
explicitně definovanou velikostí.

\begin{eqnarray}
\label{rovnice1}
x& = &\left[\Big\{\big[a+b\big]*c\Big\}^{d}\oplus{2}\right]^{3/2}\\
y& = &\lim_{x\rightarrow\infty}\frac{\frac{1}{\log_{10}x}}{\sin^{2}x+\cos^{2}x}\nonumber
\end{eqnarray}

V této větě vidíme, jak vypadá implicitní vysázení limity $\lim_{n\rightarrow\infty}f(n)$ v~normálním odstavci textu. Podobně
je to i s dalšími symboly jako $\prod_{i=1}^{n} 2^{i}$ či $\bigcap_{A\in\mathcal{B}}A$.
V~případě vzorců $\lim\limits_{n\rightarrow\infty}f(n)$ a $\underset{i=1}{\overset{n}{\prod}} 2^{i}$
jsme si vynutili méně
úspornou sazbu příkazem \verb|\limits|.

\begin{eqnarray}
\label{rovnice2}
\int_{b}^{a} g(x)\mathrm{d}x&=&-\int\displaylimits_a^{b} f(x) \mathrm{d}x
\end{eqnarray}

\section{Matice}
Pro sázení matic se velmi často používá prostředí \verb|array|
a závorky (\verb|\left|, \verb|\right|).
$$\left(\begin{array}{ccc}
{a-b} & \widehat{\xi+\omega} & \pi\\
{\vec{\mathbf{a}}} & \overleftrightarrow{AC} & \hat{\beta}
\end{array}\right) = 1 \Longleftrightarrow \mathcal{Q} = \mathbb{R}$$

$$\mathbf{A} = \left\|\begin{array}{cccc}
a_{11} & a_{12} & \ldots & a_{1 n}\\
a_{21} & a_{22} & \ldots & a_{2 n}\\
\vdots & \vdots & \ddots & \vdots\\
a_{m 1} & a_{m 2} & \ldots & a_{m n}\end{array}\right
\|=\left|
\begin{array}{cc}
t & u\\
v & w\end{array}
\right|=tw \minus uv$$

Prostředí \verb|array| lze úspěšně využít i jinde.

$$\binom{n}{k}=
\left\{\begin{array}{cl}
0 & \text{pro } k<0 \text{ nebo } k>n\\
\frac{n!}{k!(n-k)!} & \text{pro } 0 \leq{k}\leq{n.}
\end{array}\right.$$
\end{document}
