\documentclass[11pt,a4paper]{article}
\usepackage[left=2.0cm, top=3.0cm, text={17cm, 24cm}]{geometry}
\usepackage[utf8]{inputenc}
\usepackage[IL2]{fontenc}
\usepackage[czech]{babel}
\usepackage{url}

\begin{document}
\begin{titlepage}
\begin{center}
\Huge\textsc{Vysoké učení technické v~Brně}\\
\huge\textsc{Fakulta informačních technologií}\\
\vspace{\stretch{0.382}}
\LARGE{Typografie a~publikování -- 4. projekt}\\
\Huge{Bibliografické citácie}
\vspace{\stretch{0.600}}
\end{center}
{\Large{\today\hfill
Adrián Bobola}}
\vspace{\stretch{0.025}}
\end{titlepage}

\section*{Úvod do typografie}
    Písmo je základným dorozumievacím prostriedkom ľudstva už od nepamäti. Maľby v~jaskyniach
    nasvedčujú tomu, že už aj naši dávni predkovia používali písmo na
    komunikáciu.\cite{Printmag}
    \cite{typografia_1976}.
    
    Samotný termín \verb|typografia| označuje grafickú úpravu tlačených dokumentov.
    Zaoberá sa použitím vhodného rezu, typu písma a prevedením textu do vhodnej estetickej
    podoby~\cite{Fiala}.
    
\section*{Nástroj \LaTeX}    
  \LaTeX\ je softvérový balíček, ktorý nám umožňuje formátovanie textu pomocou makier a tým zjednodušuje používanie systému pre sadzbu dokumentov tzv. TeX. Bližšie informácie o~tomto systéme viď~\cite{wiki_tex}.
  
  Používanie softvéru môže byť pre niektorých užívateľov pomerne náročné. 
  Úplným začiatočníkom môže pomôcť podrobný sprievodca systémom \LaTeX\
  viď~\cite{Kopka}. 
  
\section{Matematické výrazy}
Systém \LaTeX\ je vhodný aj pre náročných používateľov, ktorý požadujú
rôzne symboly gréckej abecedy napr. $\omega, \epsilon, \lambda$, prípadne iné matematické symboly~\cite{Gratzer}. 

\section{Diagramy jazyka MSC}
Tento systém taktiež umožňuje vkladať rôzne diagramy jazyka MSC do dokumentov. Tejto téme sa venoval aj jeden zo študentov vo svojej bakalárskej práci viď~\cite{Fabry}.

\section{Dostupné balíčky}
Použitie systému \LaTeX\  nám taktiež uľahčí prácu napríklad pri
sadzbe štruktúr napríklad algoritmov, grafov a pod.~\cite{Simpson}.
Pre tento systém existuje celá škála rôznych balíčkov,
ktoré nám umožnia používať rôzne objekty, ktoré nie sú súčasťou
samotného systému.

\section{Bakalarské a diplomové práce}
Mnoho študentov používa tento systém aj na písanie rôznych záverečných prác. Niektorí užívatelia tvrdia, že je systém \LaTeX\ príliš
náročný na používanie a radšej preferujú iné nástroje~\cite{OpenWetWare}. Tento pocit je však veľmi subjektívny
a nemožno sa ním úplne riadiť.

\section{Prezentácie}
Tento nástroj taktiež môžete využiť dokonca aj pri tvorbe rôzných
prezentácii. O~zásadách správnej prezentácie sa môžete dočítať
v~\cite{Yaffe}.

\newpage
\bibliographystyle{czechiso}
\renewcommand{\refname}{Použitá literatura}
\bibliography{proj4}

\end{document}